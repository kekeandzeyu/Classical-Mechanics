\section{Lecture 3: Action Principle \& Calculus of Variations}

\subsection{Action Principle}

For example, let's take a look at the free particle moving in a circle.

We have the following equations of motion:

\begin{align*}
    x&=R\cos(\theta)\\
    y&=R\sin(\theta)\\
\end{align*}

\begin{align*}
    \dot{x}&=-R\sin(\theta)\dot{\theta}\\
    \dot{y}&=R\cos(\theta)\dot{\theta}\\
\end{align*}

\[
    L = T = \frac{1}{2} m (\dot{x}^2+\dot{y}^2) = \frac{1}{2} m R^2 \dot{\theta}^2
\]

So $L$ is independent of $\theta$.

\[
    p_\theta =  m R^2 \dot{\theta} = \text{angular momentum}
\]

\begin{definition}[Fermat's Principle]
    Light travels along the shortest path between two points.
\end{definition}

Precisely the length of a curve is given by $L=\int ds$. It will be minimized for the trajectory of light.

Does every mechanical system have a principle like this? Yes, it is called the \textbf{Action Principle}.

This is called the \textbf{Least Action Principle}, or \textbf{Hamilton's Principle}.

The path that the system takes is the one that "minimizes" $S[q_i(t)]$.