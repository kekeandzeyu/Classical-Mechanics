\section{Lecture 3: Action Principle \& Calculus of Variations}

\subsection{Action Principle}

For beginning, let's recall the Fermat's Principle.

\begin{definition}[Fermat's Principle]
    Light travels along the shortest path between two points.
\end{definition}

Precisely speaking, the length of a curve $L = \int ds \sqrt{\left(\frac{dx}{ds}\right)^2 + \left(\frac{dy}{ds}\right)^2}$
will be minimized for the trajectory of light.

Then we need to ask the following question: does every mechanical system obey a 
minimization principle of this sort? The answer is yes. So let's consider the set of all
possible paths $q_i\left(t\right)$ that a system could take through configuration space.

For a given path $q_i\left(t\right)$, we define the action of this path as
$S[q_i\left(t\right)]$.

\[
    s[q_i\left(t\right)] = \int_{t_\text{initial}}^{t_\text{final}} L\left(q_i, \dot{q}_i, t\right) dt
\]

The path that a mechanical system takes through configuration space (nearly space of $q_i$)
"minimizes" $S[q_i\left(t\right)]$ (not global, only local extremum). This is called the
\textbf{Least Action Principle}, or alternatvely, \textbf{Hamiltonian's Principle}. 
$S[q_i\left(t\right)]$ is a function of a function, so it is called a \textbf{functional}.
For functionals, we use square brackets $[f]$ to denote the dependence rather than curve bracket
$f\left(x\right)$.

We are used in single (or multi) variable calculus to minimizing a function of 1(N) variables.

We need to minimize a functional, a function of $\infty$ number of variables. 